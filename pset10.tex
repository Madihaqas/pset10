\documentclass[a4paper]{exam}

\usepackage{geometry}
\usepackage{amsmath}
\usepackage{amsfonts}

\newcommand\Z{\ensuremath{\mathbb{Z}}}
\newcommand\R{\ensuremath{\mathbb{R}}}
\newcommand\Q{\ensuremath{\mathbb{Q}}}

\title{Problem Set 10: Cardinality Of Sets}
\author{CS/MATH 113 Discrete Mathematics}
\date{Spring 2024}

\boxedpoints

\printanswers

\begin{document}
\maketitle

\begin{questions}
\question 
  Show that the two given sets have equal cardinality by describing a bijection from one to the other. Describe your bijection with a formula (not as a table). You need not include a proof of bijection.
  \begin{parts}
  
  \part The set of even integers and the set of odd integers
    \begin{solution}
    
    $f$ : $E \implies O$\\
    The function can be written as:\\
    $f(n)$ : $n-1$ where n is any even integer.\\
    Claim $f(n)$ is injective:\\
    Suppose $f(a)= f(b)$\\
     $a-1 = b-1$ \\
     $a = b$\\
     This shows that the function is 1 to 1.\\
     Claim $f(n)$ is surjective:\\
     $f(n)$ : $n-1$\\
     $f(y)$ = $x-1$\\
     $x = y+1$\\
     Hence the function is onto.\\
     We can conclude that the function is bijective.     
    \end{solution}
    
  \part $\Z$ and $S = \{..., \frac{1}{8}, \frac{1}{4}, \frac{1}{2}, 1, 2, 4, 8, 16, ...\}$
    \begin{solution}
    
      set $S$ can also be written as S=\{2n:n$\in$ Z\}\\
      $f: Z \implies S$\\    
      $f(x)=2^x$\\
    Because set S follows a sequence, we can say that S has the same cardinality as N. We also know that Z has the same cardinality as N. Therefore, both Z and S have the same cardinality, which implies that a bijection between the two exists.
    \end{solution}

  \part $\Z^+$ and $S = \{2^n : n \in \Z^+\}$
    \begin{solution}
    
      S=\{2,4,8,26,32..\}\\
      Since members of set S follow a sequence, we can say that the cardinality of set S is equal to the cardinality of N.\\
      We know that there exists a bijection between $Z+$ and N.\\
      Therefore we can conclude that $Z+$ and S have equal cardinality 
    \end{solution}
    
  \part $A = \{3k : k \in \Z\}$ and $B = \{7k : k \in \Z\}$
    \begin{solution}
      
     Set A is the set of all integers multiplied by 3, and set B is the set of all integers multiplied by 7. We know that $Z$ has the same cardinality as $N$; therefore, both set A and set B will have equal cardinalities, implying that a bijection exists between the two sets.
    \end{solution}
    
  \part $\R$ and $S=\{x\in\R\mid x>0 \}$
    \begin{solution}
      
       $f: R \implies S$\\
      $f(x)=x+1$\\
      To prove that the function is bijective, we need to prove that it is both Injective and Surjective:\\
      Claim 1: the function is injective:\\
      Consider $x1 and x2$\\
      $f(x1)=x1+1$ and $f(x2)=x2+1$\\
       $f(x1)=x1 = f(x2)=x2$\\
       Hence proved that the function is injective.\\
       Claim 2: the function is surjective:\\
         $y=x+1$\\
          $y-1=x$\\
        Hence proved that the function is surjective.\\
        Therefore the sets are bijective and have the same cardinality
      
    \end{solution}
    
  \part $\R$ and $S=\{x\in\R\mid x>\sqrt{2} \}$
    \begin{solution}
      
      $f: R \implies S$\\
      $f(x)=x+\sqrt{2}$\\
      To prove that the function is bijective, we need to prove that it is both Injective and Surjective:\\
      Claim 1: the function is injective:\\
      Consider $x1 and x2$\\
      $f(x1)=x1+\sqrt{2}$ and $f(x2)=x2+\sqrt{2}$\\
       $f(x1)=x1 = f(x2)=x2$\\
       Hence proved that the function is injective.\\
       Claim 2: the function is surjective:\\
         $y=x+\sqrt{2}$\\
          $y-\sqrt{2}=x$\\
        Hence proved that the function is surjective.\\
        Therefore the sets are bijective and have the same cardinality
      
    \end{solution}
  \end{parts}

\question Prove or disprove each of the following statements.
  \begin{parts}
  \part If $A = \{X \subseteq \Z^+ \mid X \text{ is finite}\}$, then $|A| = \aleph_0$.
    \begin{solution}
    
      Following direct proof:\\
      We know that $X$ is the subset of positive integers and $X$ is finite.\\
      Therefore the cardinality of set $A$ must be less then that of a countably infinite set.\\
      then $|A| < \aleph_0$.
      
    \end{solution}
  \part The set $A = \{(m,n) \in \Z^+ \times \Z^+ \mid m \leq n\}$ is countably infinite.\\
    \begin{solution}
     
      Because the elements of the set can be written in a sequence and follow a pattern, we can say that the set A is countably infinite.
    \end{solution}
  \part The set $\Z \times \Q$ is countably infinite.
    \begin{solution}
     
      Because $Z$ is countably infinite and $Q$ is countably infinite, their cartesian product will also be countably infinite.
    \end{solution}

  \part If $A \subseteq B$ and there is an injection $g : B \to A$, then $|A| = |B|$.
    \begin{solution}
      
      From the definition of injection we know that each element of B is mapped to exactly one unique element of A, A is the subset of B, if B's element are all mapped to A then that means that both A and B have the same cardinality.
    \end{solution}
    
  \part  If $|A| = |B|$ and $|B| = |C|$, then $|A| = |C|$.
    \begin{solution}
      
      Using the transitive property of equality, we know that the cardinality of A is equal to B, and cardinality of B is equal to C, we can conclude that the cardinality of A is in fact equal to C. 
    \end{solution}
    
  \part If $A$ and $B$ are sets with $|A| = |B|$, then $|\mathcal{P}(A)| = |\mathcal{P}(B)|$.
    \begin{solution}
     
      $A$ and $B$ contain the same numbers of elements, then the number of subsets will also be equal which implies that their Power sets will also have the same cardinality.
    \end{solution}
  \end{parts}
  
\end{questions}
\end{document}
%%% Local Variables:
%%% mode: latex
%%% TeX-master: t
%%% End:
